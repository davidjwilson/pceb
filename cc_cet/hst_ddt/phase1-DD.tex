%%%%%%%%%%%%%%%%%%%%%%%%%%%%%%%%%%%%%%%%%%%%%%%%%%%%%%%%%%%%%%%%%%%%%%%%%%
%
%    phase1-DD.tex  (use only for General Observer DD proposals; use phase1-AR.tex for Archival Research
%                      and Theory proposals and use phase1-GO.tex for General Observer, Snapshot and Survey proposals
%                      and use phase1-MC.tex for GO/MC rapid response proposals).
%
%    HUBBLE SPACE TELESCOPE
%    PHASE I GO DD PROPOSAL TEMPLATE 
%
%    Version 2.1; November 2018
%
%    Guidelines and assistance
%    =========================
%    Cycle 27 Announcement Web Page:
%
%         http://hst-docs.stsci.edu/
%
%    Please contact the STScI Help Desk if you need assistance with any
%    aspect of proposing for and using HST. Either send e-mail to
%    help@stsci.edu, or call 1-800-544-8125; from outside the United
%    States, call [1] 410-338-1082.
%
%%%%%%%%%%%%%%%%%%%%%%%%%%%%%%%%%%%%%%%%%%%%%%%%%%%%%%%%%%%%%%%%%%%%%%%%%%%

% The template begins here. Please do not modify the font size from 12 point.

\documentclass[12pt]{article}
\usepackage{phase1}

\begin{document}
%   1. RATIONALE FOR DD TIME
%
%       Explain why DD time is required; i.e., why the proposal was not submitted
%       to the most recent TAC, or explain why the proposal cannot wait until the
%       next TAC for evalution.
%       
%
\rationaletime          % Do not delete this command.
% Enter your rationale for DD time here. 
ENTER YOUR TEXT HERE.

%%%%%%%%%%%%%%%%%%%%%%%%%%%%%%%%%%%%%%%%%%%%%%%%%%%%%%%%%%%%%%%%%%%%%%%%%%%

%   2. SCIENTIFIC JUSTIFICATION
%
%      This section should present a balanced discussion of background information,
%      the program's goals, its significance to astronomy in general, and its importance 
%      for the specific subfield of astronomy that it addresses. 
%
\justifyscience         % Do not delete this command.
% Enter your Science Justification here.
ENTER YOUR TEXT HERE.

%%%%%%%%%%%%%%%%%%%%%%%%%%%%%%%%%%%%%%%%%%%%%%%%%%%%%%%%%%%%%%%%%%%%%%%%%%%

%   3. DESCRIPTION OF THE OBSERVATIONS
%
%       Provide a description of the proposed observations; Explain the 
%       amount of exposure time and number of orbits requested (e.g., number 
%       of objects, examples of exposure-time calculations and orbit estimations 
%       for some typical observations, etc.). Explicitly describe any 
%       non-standard calibration requirements and observations.
%
\describeobservations   % Do not delete this command.
% Enter your observing description here.
ENTER YOUR TEXT HERE.

%%%%%%%%%%%%%%%%%%%%%%%%%%%%%%%%%%%%%%%%%%%%%%%%%%%%%%%%%%%%%%%%%%%%%%%%%%%

%   4. SCHEDULING REQUIREMENTS
%
%       Provide any special scheduling requirements (such as required and 
%       desired execution windows, special orientation or background 
%        requirements, and time links to HST or other observations) to allow 
%        for scheduling impact assessment. For minimum schedule disruption, 
%        STScI requires that all observing information be submitted at least 2 
%        months prior to execution, in cases where this is possible. 
%
\schedulingreqs             % Do not delete this command.
% Justify your scheduling requirements here, if any.
ENTER YOUR TEXT HERE.

%%%%%%%%%%%%%%%%%%%%%%%%%%%%%%%%%%%%%%%%%%%%%%%%%%%%%%%%%%%%%%%%%%%%%%%%%%%

\end{document}          % End of proposal. Do not delete this line.
                        % Everything after this command is ignored.